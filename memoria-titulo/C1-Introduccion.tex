\section{Introducción}

\subsection{Contexto}
El contexto, abarcando ya sea la pregunta de investigación o la declaración del problema, sirve como la base de una memoria o tesis. Se define el alcance, la importancia y la orientación de la investigación, guiando todo el estudio. Crear un contexto convincente y claro es crucial, ya que no solo informa a los lectores sobre el propósito de la investigación, sino que también establece la necesidad y relevancia de la investigación.

Se comienza resumiendo el estado actual del conocimiento en su campo de estudio (estado del arte). Aquí, se destacan las investigaciones existentes y se señalan las brechas o preguntas sin responder que el estudio pretende abordar. Este enfoque no solo prepara el escenario para la investigación, sino que también demuestra su necesidad y contribución potencial al campo.

Después de establecer la brecha, se articula la pregunta de investigación o declaración del problema a resolver según sea el caso. Esto debería abordar directamente la brecha identificada, especificando lo que la investigación investigará o resolverá. Uno debe asegurarse que la pregunta o problema sea enfocado, investigable y significativo para el campo de estudio.

Luego, se debe explicar por qué la pregunta de investigación o problema es importante de abordar. Se discuten las implicaciones potenciales de su estudio y cómo podría contribuir al campo o aplicarse a problemas del mundo real. Esta justificación establece el valor y la contribución del estudio. Poner cuidado en que la pregunta de investigación o declaración del problema sea lo suficientemente específica para ser manejable dentro del alcance de la memoria o tesis. 

Al elaborar el contexto, el foco es involucrar a los lectores presentando una narrativa clara y convincente que destaque la importancia de su investigación y cómo llena una brecha específica en el estado del arte. Este elemento fundamental establece el tono para todo el estudio, guiando la dirección de la investigación e informando su metodología y análisis.

\subsection{Objetivos}
\lipsum[4]

\subsection{Hipótesis}
\lipsum[5-6]

\subsection{Descripción del trabajo}
\lipsum[7-9]




